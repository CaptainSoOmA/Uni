\subsection*{Aufgabe 1}

Erläutern Sie die Funktionsweise binärer Semaphore zur Sicherstellung des gegenseitigen
Ausschlusses mehrerer Prozesse an folgendem Beispiel, wobei Sie den jeweiligen Zustand
in Tabellenform protokollieren:

Initialisierung mit 1; 
Prozess 1 ruft P auf; 
Prozess 2 ruft P auf; 
Prozess 3 ruft P auf;
Prozess 1 ruft V auf; 
Prozess 4 ruft P auf; 
Prozess 3 ruft V auf; 
Prozess 5 ruft P auf.

\paragraph*{Tabelle:}

\begin{tabular}{ c | c | c | c }
	Prozess  & Aufruf &  Semaphore & Bemerkung\\
	\hline
		&	I 	& 	1		&	Initialisierung		\\
	1	&	P	&	0 		& 	1. im k.A.		\\
	2	&	P	&	-1		&	2 blockiert		\\
	3	&	P	&	-2		& 	3 blockiert		\\
	1	&	V	&	-1		&	2. im k.A.		\\
	4	&	P	&	-2		&	4 blockiert		\\
	3	&	V	&	-2		&	3. im k.A.		\\
	5	&	P	&	-3		&	5 blockiert		\\
	
\end{tabular}

