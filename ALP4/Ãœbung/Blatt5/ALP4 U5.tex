\documentclass[11pt]{scrartcl} %for standart A4: a4paper as option and remove package a4wide (scrartcl is the KOMA-class of article)
\usepackage{a4wide} %A4 document size with smaller margins
\usepackage[utf8]{inputenc} %Text Encoding
\usepackage[ngerman]{babel} %new German orthography
\usepackage{totpages} %\ref{TotPages} for total page count
\usepackage{graphicx}

% math packages 
\usepackage{amsmath, amsfonts, amssymb, amsthm}

% Hyperlinks 
\usepackage{hyperref}

\usepackage{listings}

%------------------------------------------------------------------------------
\newcommand{\UEBUNGSNR}{5}	

\date{\today}
\author{Tobias Kranz (414 71 30)\\ Johannes Rohloff ()}
\title{ALP4 - Nichtsequentielle Programmierung \UEBUNGSNR{}}
%------------------------------------------------------------------------------

%paragraph handling (e.g. bigger gap between them) 
\usepackage{parskip}

%header/footer
\usepackage{fancyhdr}
\setlength{\headheight}{15.2pt}
\pagestyle{fancyplain}
	\lhead{ALP4 - Ü\UEBUNGSNR{}}
	\chead{Tutor: Julian Fleischer (Mi 14-16)}
	\rhead{T. Kranz, J. Rohloff}
	\lfoot{}
	\cfoot{Seite \thepage\ von \ref{TotPages}}
	\rfoot{}
	\renewcommand{\headrulewidth}{0.4pt}
	\renewcommand{\footrulewidth}{0.4pt}
\begin{document}
\maketitle
\paragraph{Aufgabe 1}

\subparagraph{a)} Prüfen auf Zyklen ist mittels Tiefensuche möglich.

\subparagraph{b)} Die Laufzeit beträgt, wenn der Graph als Adjazensliste gespeichert wird, $\theta$(\#Knoten + \#Kanten) (lineare Laufzeit) bzw. $\theta$(\#Knoten$^2$) (quadratische Laufzeit) sofern er als Adjazensmatrix vorliegt.

\paragraph{Aufgabe 2}  

\subparagraph{a)}\

\lstinputlisting[language=C]{LS-Monitor.c}

\subparagraph{b)}\

\lstinputlisting[language=C]{LS-Monitor.go}
\end{document}


\paragraph{Aufgabe 4}  
