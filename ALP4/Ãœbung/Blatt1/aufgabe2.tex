\subsection*{Aufgabe 2}

Aus der vorlesung wissen wir:

\begin{equation*}
  \frac{(n_{1} + n_{2} + ... + n_{p} )!}
  {n_{1}! \cdot n_{2}! \cdot ... \cdot n_{p}! }
\end{equation*}
bei p folgen aus $n_{1} , n_{2} , ... , n_{p} $ Anweisungen

\paragraph{a)}
\paragraph{Beispiel 1:} 2 Folgen aus je 3 Anweisungen
p = 2
$ n_{1} = n_{2} = 3$

\begin{equation*}
  \frac{(3 + 3)!}
  {3! \cdot 3! }
  =
  \frac{(6)!}
  {3 \cdot 3!}
  = 
  \frac{4 \cdot 5 \cdot 6}
  {2 \cdot 3}
  = 4 \cdot 5
  = 20
\end{equation*}


\paragraph{Beispiel 2:} 4 Folgen aus je 4 Anweisungen über 60 Millionen
p = 4
$ n_{1} = n_{2} = n_{3} = n_{4} = 4$

\begin{equation*}
  \frac{(4 + 4 + 4 + 4)!}
  {4! \cdot 4! \cdot 4! \cdot 4! }
  =
 \frac{(16)!}
  {4! \cdot 4! \cdot 4! \cdot 4! }
  = 16 ! / 96
  = 63063000
  = 63 * 10^{6}
\end{equation*}


\paragraph{Beispiel 3:} 6 Folgen aus je 5 Anweisungen bereits 90 Trillionen
p = 6
$ n_{1} = ...= n_{6} = 5$

\begin{align*}
  \frac{(5 + 5 + 5 + 5 + 5 +5)!}
  {5! \cdot 5! \cdot 5! \cdot 5! \cdot 5! \cdot 5!}
  &=\frac{30)!}
  {5! \cdot 5! \cdot 5! \cdot 5! \cdot 5! \cdot 5!}\\
  &= 88832646059788350720\\
  &= 88.83 * 10^{18}
\end{align*}


\paragraph{Beispiel 4:}8 Folgen aus je 6 Anweisungen 
p = 8
$ n_{1} = ...= n_{8} = 6$

\begin{align*}
  \frac{(6 + 6 + 6 + 6 + 6 +6 + 6 +6 +6)!}
  {6! \cdot 6! \cdot 6! \cdot 6! \cdot 6! \cdot 6! \cdot 6! \cdot 6!}
  &=\frac{48)!}
  {6! \cdot 6! \cdot 6! \cdot 6! \cdot 6! \cdot 6! \cdot 6! \cdot 6!} \\
  &= 171889289584866507880743491472699801600\\
  &= 171.88 * 10^{36}
\end{align*}



\paragraph{b)}



für die gegebenen Werte folgt:

\begin{align*}
  \frac{(10 + 10 + ... + 10 )!}
  {10! \cdot 10! \cdot ... \cdot 10! }
  &=
  \frac{(100)!}
  {10! \cdot 10! \cdot ... \cdot 10! }\\
  &= 235707458939304389640931968316130209128979624196658578574141046497349714005349706689167360000\\
  &= 2.35 * 10^{92}
\end{align*}

Geht man von 100 Anweisungen pro seite aus so erhällt man:

\begin{equation*}
  2.35 * 10^{92} / 100
  == 2.35 * 10^{90} Seiten
\end{equation*}

ein Blatt hat ein Volumen von $ 3.6 * 10^{-8}m^3$. Somit ergibt sich ein gesamtes Volumen von:
\begin{equation*}
 2.35 * 10^{90} Seiten = 2.35 * 10^{90} * 3.6 * 10^{-8}m^3 = 9.1 * 10^{88}
\end{equation*}



Das Universum hat eine Ausdehnung von

\begin{equation*}
  93 * 10^{9} Lj
  = 93 * 10^9 * 9.461 * 10^{15} m
  = 879.873 *  10^{24}
  = 8.79 * 10^{26}
\end{equation*}

Ein Kugel mit dem Volumen des errechneten hat folgenden dutchmesser:
\begin{equation*}
  V = 3/4 \pi * r^3
\end{equation*}
\begin{equation*}
  9.1 * 10^{88} = 3/4 \pi * r^3
\end{equation*}
\begin{equation*}
 r^3 = 9.1 * 10^{88}* 4/3  /\pi 
\end{equation*}
\begin{equation*}
 r = \sqrt[3]{9.1 * 10^{88}* 4/3  /\pi } = \sqrt[3]{3.862 10^{88}} = 3.38 * 10^{29}
\end{equation*}

Die Aussage stimmt so nicht. Die Kugel ist um den Faktor 1000 mal größer!!
